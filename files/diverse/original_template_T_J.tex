\documentclass[11pt]{scrartcl}

\usepackage{graphicx}
\usepackage{lastpage}
\usepackage{amsmath}
\usepackage{amssymb}
\usepackage{fancyhdr}

\usepackage[labelsep=colon,justification=justified,labelfont=bf,textfont=sl]{caption,subcaption}
\usepackage[utf8]{inputenc}
\usepackage[ngerman]{babel}
\usepackage[margin=3cm,a4paper]{geometry}
\usepackage{enumitem} %für Nummerierung bei enumerate
%Bsp.: \begin{enumerate}[label=(\roman*)]
\usepackage{gensymb} %eigentlich nur für ° da,,
%Ok mit dem geht auch \celius für °C

%Tabellen und Bilder
\captionsetup[table]{name=Tabelle }
\captionsetup[figure]{name=Abbildung }

%Kopf & Fuß
\pagestyle{fancy}
\setlength{\headheight}{25.3pt}
\lhead{Versuchsname}
\chead{Name1\\ Name2}
\rhead{dd.mm.yyyy}
\cfoot{\thepage \hspace{1pt} / \pageref{LastPage}}

%Neue Namen
\renewcommand{\contentsname}{Inhaltsangabe}
\renewcommand{\listtablename}{Tabellenverzeichnis}
\renewcommand{\listfigurename}{Abbildungsverzeichnis}

%Bilder Directories
%\graphicspath{{{Plots/}}

%Eigene Commands
\newcommand{\der}[2]{\frac{\mathrm{d}#1}{\mathrm{d}#2}}
\newcommand{\pder}[2]{\frac{\partial #1}{\partial #2}}

\begin{document}
\tableofcontents
\newpage

\section{Aufgabenstellung\label{Auf0}}

Siehe:

\section{Vorbereitung}

Siehe:

\section{Grundlagen \& Berechnungen}
%Gleichungen werden so oder mit \begin{equation} formatiert
\begin{align}
    \nabla \cdot \textbf{E}  & = \frac{\rho}{\varepsilon_0}                                                              \\
    \nabla \cdot \textbf{B}  & = 0                                                                                       \\
    \nabla \times \textbf{E} & = \frac{\partial \textbf{B}}{\partial t}                                                  \\
    \nabla \times \textbf{B} & = \mu_{0}\left(\textbf{J} + \varepsilon_{0} \frac{\partial \textbf{B}}{\partial t}\right) \\
\end{align}

$$\lim_{n\to\infty} (\frac{1}{2})^n$$
$\frac{1}{2}$
\section{Versuchsdurchführung \& Messergebnisse}
Hello Test Miau
\subsection{Geräteliste}
%Beispiel Tabelle 
\captionof{table}{Verwendete Geräte}
\begin{center}
    \begin{tabular}{|c|c|c|c|c|} \hline
        \textbf{Gerät} & \textbf{Modell} & \textbf{Hersteller} & \textbf{Nr.} & \textbf{Us} \\ \hline
        Der Gerät      & Nie müde        & Schweißfrei Inc.    & 12           & $10^{-42}$  \\ \hline
    \end{tabular}
\end{center}

\section{Auswertung}

\section{Diskussion \& Zusammenfassung}

%\section*{Literatur}
%\listoffigures
%\listoftables
\end{document}
